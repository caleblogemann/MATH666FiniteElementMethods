\documentclass[11pt, oneside]{article}
\usepackage[letterpaper, margin=2cm]{geometry}
\usepackage{booktabs}
\usepackage{Logemann}
\usepackage{Integral}
\usepackage{LinearAlgebra}
\usepackage{Derivative}
\usepackage{Vector}
\usepackage{Sum}
\usepackage{SetTheory}
\usepackage[backend=biber]{biblatex}
\addbibresource{refs.bib}

\title{Discontinuous Galerkin Method for Solving Thin Film Equations}
\author{Caleb Logemann}

\date{December 13, 2018} % Date, can be changed to a custom date

\begin{document}
\maketitle

\begin{abstract}
  This paper will describe a discontinuous Galerkin method for solving
  thin film equations. The model equation is
  $q_t + \p{q^2 - q^3}_x = -\p{q^3 q_{xxx}}_x$.
  This equation will be handled using operator splitting.
  An explicit Runge Kutta discontinuous Galerkin method will be used to solve the
  convection term.
  An implicit local discontinuous Galerkin method will solve the fourth order
  diffusion equation.
  Furthermore a preconditioned linear solver for the implicit method will be
  described.
\end{abstract}

\section{Introduction}
  Thin film equations arise in several applications involving lubrication theory.
  For example thin films of water flowing over an airplane wing during flight.
  When an airplane flies in wet conditions ice can begin to build up on the
  surface of the plane.
  An important part of modeling the accretion of ice on a plane is to study how
  the water might move over the surface of the plane before freezing.
  This is known as runback.
  Thin film equations also arrive in different industrial applications, such as
  industrial coating.

  In these contexts several forces drive the motion of the fluid, including
  gravity, wind shear, and surface tension.
  Beginning with the Navier-Stokes equations shown below,
  \begin{align*}
    \nabla \cdot \v{u} &= 0 \\
    \partial_t \v{u} + \nabla \cdot \p{\v{u}\v{u}} &= - \frac{1}{\rho} \nabla p + \frac{1}{\rho}\nabla \cdot \sigma + \v{g} \\
    \partial_t h_s + \p{u, v}^T \cdot \nabla h_s &= w \\
    \partial_t h_b + \p{u, v}^T \cdot \nabla h_b &= w
  \end{align*}
  and using a lubrication approximation to model these driving forces the model
  equation
  \[
    q_t + \p{q^2 - q^3}_x = -\p{q^3 q_{xxx}}_x
  \]
  can be derived.
  See the review by Myers\cite{myersReview} for a derivation.

\section{Method}
  The model equation will be solved using operator splitting, the convection
  equation
  \[
    q_t + \p{q^2 - q^3}_x = 0
  \]
  and the diffusion equation
  \[
    q_t + \p{q^3 q_{xxx}}_x = 0
  \]
  will be solved consecutively using different methods in order to
  approximate the solution to the original equation.
  In order to achieve 2nd order accuracy Strang splitting can be used.
  In Strang splitting one whole time step, $\Delta t$ consists of stepping the
  convection equation $\frac{1}{2}\Delta t$, the diffusion equation $\Delta t$,
  and the convection equation again by $\frac{1}{2}\Delta t$.

  The discontinuous Galerkin method will be used to solve both of these
  equations, so first some notation will be introduced.
  The domain of this problem $\Omega$ will be some interval on the real line,
  $\br{a, b}$.
  Let this domain be partitioned by $N+1$ evenly spaced points
  $a = x_{1/2} < x_{3/2} < \cdots < x_{i+1/2} < \cdots < x_{N+1/2} = b$.
  Let $h = x_{i+1/2} - x_{i-1/2}$ be the distance between points and let
  $x_i = \frac{1}{2}\p{x_{i-1/2} + x_{i+1/2}}$ be cell centers.
  This partition will create $N$ intervals denoted
  $I_j = \br{x_{j-1/2}, x_{j+1/2}}$.
  The DG space can then be defined as
  $V_h = \set{v \in L^2(\Omega): \eval{v}{I_j} \in P_k(I_j)}$, where $P_k(I_j)$
  is the set of all  polynomials of order $k$ or less.
  In this case the spatial order of the DG method will be $k+1$.
  On a single interval the canonical variable $\xi$ may be used, such that for
  $x \in I_j$, then $\xi \in \br{-1, 1}$.
  This transformation can be defined as $x = x_j + \frac{1}{2}h\xi$.
  When using this variable the model equations become
  \[
    q_t + \frac{2}{\Delta x} \p{q^2 - q^3}_\xi = 0
  \]
  and
  \[
    q_t + \frac{16}{\Delta x^4} \p{q^3 q_{xxx}}_x = 0
  \]

\subsection{Convection Equation}
  The convection equation will be handled using the standard Runge Kutta
  discontinuous Galerkin method.
  This method can be described as finding a function $Q(t, x)$ such that for
  every $t > 0$, $Q(t, \cdot) \in V_h$ and that satisfies
  \begin{align*}
    \dintt{I_j}{}{Q_t v}{x} &= \dintt{I_j}{}{f(Q)v_x}{x} - \p{\mcF_{j + 1/2}v^-(x_{j+1/2}) - \mcF_{j - 1/2}v^+(x_{j-1/2})}
  \end{align*}
  for all $v \in V_h$, where the numerical flux is the local Lax-Friedrichs flux
  given by
  \[
    \mcF_{j+1/2} = \frac{1}{2}\p{f\p{Q^-_{j+1/2}} + f\p{Q^+_{j+1/2}}} + \max[q]{\abs{f'(q)}}\p{Q^-_{j+1/2} - Q^+_{j+1/2}}.
  \]
  If the Legendre polynomials, $\phi^{\ell}$ are used as a basis for the DG space on each
  element, then the semi-discrete form of this method is
  \[
    \dot{Q_i^{\ell}} = \frac{1}{\Delta x}\dintt{-1}{1}{f(Q_i)\phi_{\xi}^{\ell}}{\xi} - \frac{1}{\Delta x} \p{\mcF_{i + 1/2}\phi(1) - \mcF_{i - 1/2}\phi(-1)}
  \]
  where
  \[
    \mcF_{j+1/2} = \frac{1}{2}\p{f\p{Q_{i+1}(-1)} + f\p{Q_{i}(1)}} + \max[q]{\abs{f'(q)}}\p{Q_{i}(1) - Q_{i+1}(-1)}.
  \]

  This system of ordinary differential equations can now be solved using any ODE
  solver.
  I will denote this system at
  \[
    \dot{\v{Q}} = L(\v{Q}).
  \]
  For a first order solver, the forward Euler method will be used
  \[
    \v{Q}^{n+1} = \v{Q}^n + L(\v{Q}^n)
  \]
  and for a second order solver the following Runge Kutta method will
  be used
  \begin{align*}
    \v{Q}^{\star} &= \v{Q}^n + \Delta t L(\v{Q}^n) \\
    \v{Q}^{n+1} &= \frac{1}{2}\p{\v{Q}^n + \v{Q}^{\star}} + \frac{1}{2} \Delta t L(\v{Q}^{\star}).
  \end{align*}
  Both of these time stepping methods are strong stability preserving (SSP),
  that is they are total variation diminishing.

\subsection{Diffusion Equation}
  The local discontinuous Galerkin (LDG) method will be used to solve the fourth order
  diffusion equation. 
  The fourth order diffusion equation
  \begin{align*}
    q_t + \p{q^3 q_{xxx}}_x &= 0
  \end{align*}
  will be rewritten in the following form by introducing auxilliary variables
  \begin{align*}
    r &= q_{x} \\
    s &= r_{x} \\
    u &= q^3 s_{x} \\
    q_t &= -u_{x}.
  \end{align*}
  In order to linearize this system, let $\eta = \p{Q^n}^3$, that is use the
  value of $Q$ at the previous time step as a constant function.
  This will allow the a linear system of ODEs to be formed instead of being
  nonlinear in $Q$.
  The LDG method can then be described as finding $Q(t, \cdot), R, S, U \in V_h$
  such that
  \begin{align*}
    \dintt{I_j}{}{R v}{x} &= -\dintt{I_j}{}{Q v_x}{x} + \p{\hat{Q}_{j+1/2}v^-_{j+1/2} - \hat{Q}_{j-1/2} v^+_{j-1/2}} \\
    \dintt{I_j}{}{S w}{x} &= -\dintt{I_j}{}{R w_x}{x} + \p{\hat{R}_{j+1/2}w^-_{j+1/2} - \hat{R}_{j-1/2} w^+_{j-1/2}} \\
    \dintt{I_j}{}{U y}{x} &= \dintt{I_j}{}{S_x \eta y}{x} - \p{S^-_{j+1/2}\eta^-_{j+1/2}y^-_{j+1/2} - S^+_{j-1/2}\eta^+_{j-1/2}y^+_{j-1/2}} \\
    &+ \p{\hat{S}_{j+1/2} \hat{\eta}_{j+1/2} y^-_{j+1/2} - \hat{S}_{j-1/2} \hat{\eta}_{j-1/2} y^+_{j-1/2}} \\
    \dintt{I_j}{}{Q_t z}{x} &= -\dintt{I_j}{}{U z_x}{x} + \p{\hat{U}_{j+1/2}z^-_{j+1/2} - \hat{U}_{j-1/2} z^+_{j-1/2}}
  \end{align*}
  for all $I_j \in \Omega$ and all $v, w, y, z \in V_h$.
  In this case the numerical fluxes are the so called alternating fluxes and the
  $\eta$ flux is just an average flux.
  \begin{align*}
    \hat{\eta}_{j+1/2} &= \frac{1}{2}\p{\eta^+_{j+1/2} + \eta^-_{j+1/2}} \\
    \hat{Q}_{j+1/2} &= Q^+_{j+1/2} \\
    \hat{R}_{j+1/2} &= R^-_{j+1/2} \\
    \hat{S}_{j+1/2} &= S^+_{j+1/2} \\
    \hat{U}_{j+1/2} &= U^-_{j+1/2}
  \end{align*}


\section{Results}

\section{Conclusion}

\section{References}
  \printbibliography{}

\end{document}
