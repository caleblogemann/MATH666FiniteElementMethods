\documentclass[11pt, titlepage]{article}
\author{Caleb Logemann}
\title{MATH 666: Finite Element Methods \\ Homework 2}
\date{October 23, 2018}
\usepackage[letterpaper, margin=2cm]{geometry}
\usepackage{MATH666}
\usepackage{booktabs}

\begin{document}
\maketitle

%\lstinputlisting[language=MATLAB]{H01_23.m}
\begin{enumerate}
  \item[\#1]
    Consider the 2D Poisson equation:
    \begin{align*}
      \mathbf{PDE}:& \quad -\nabla \cdot \nabla u = f(x, y) \quad \text{in} \quad \Omega = \br{-1,1} \times \br{-1,1} \\
      \mathbf{BC}:& \quad u + \nabla u \cdot \hat{\v{n}} = g \quad \text{on} \quad  \partial\Omega
    \end{align*}
    \begin{enumerate}
      \item[(a)]
        Recast this problem as a variational problem.
        Clearly state the test and trial function spaces.

        In order to recast this as a variational problem, I will multiply by a
        test function and integrate over $\Omega$.
        \begin{align*}
          \diintt{\Omega}{}{-\p{\nabla \cdot \nabla u} v}{\v{x}} &= \diintt{\Omega}{}{f v}{\v{x}}
          \intertext{Integrating by parts gives}
          \diintt{\Omega}{}{\nabla u \cdot \nabla v}{\v{x}} - \dintt{\partial\Omega}{}{\p{\nabla u \cdot \hat{\v{n}}} v}{s} &= \diintt{\Omega}{}{f v}{\v{x}}
          \intertext{Using the boundary condition we see that $\nabla u \cdot \hat{\v{n}} = g - u$ on $\partial \Omega$}
          \diintt{\Omega}{}{\nabla u \cdot \nabla v}{\v{x}} - \dintt{\partial\Omega}{}{\p{g - u} v}{s} &= \diintt{\Omega}{}{f v}{\v{x}} \\
          \diintt{\Omega}{}{\nabla u \cdot \nabla v}{\v{x}} + \dintt{\partial\Omega}{}{uv}{s} &= \diintt{\Omega}{}{f v}{\v{x}} + \dintt{\partial \Omega}{}{gv}{s} \\
        \end{align*}
        In order for this equation to be well defined $u$ and $v$ must be in
        the space $H^1(\Omega)$, that is they are square integrable on $\Omega$
        and the norm of their gradients are square integrable on $\Omega$.
        Note that being in $H^1(\Omega)$ implies $L^2(\partial \Omega)$, so the
        boundary integrals are also well defined.

        So the bilinear form of this variational problem is to find $u \in H^1(\Omega)$ such that
        \[
          B(u, v) = L(v)
        \]
        for all $v \in H^1(\Omega)$, where
        \[
          B(u,  v) = \diintt{\Omega}{}{\nabla u \cdot \nabla v}{\v{x}} + \dintt{\partial\Omega}{}{uv}{s}
        \]
        and
        \[
          L(v) = \diintt{\Omega}{}{f v}{\v{x}} + \dintt{\partial \Omega}{}{gv}{s}.
        \]

      \item[(b)]
        Show that the variational problem has a unique solution by showing that
        it meets all of the criteria of the Lax-Milgram Theorem.

        The four conditions of the Lax-Milgram Theorem are symmetry, continuity
        of $B$, V-ellipticity, and continuity of $L$, and are shown below.
        \begin{align*}
          B(u, v) &= B(v, u) \\
          \abs{B(u, v)} &\le \gamma \norm[V]{u} \norm[V]{v} \\
          B(v, v) &\ge \alpha \norm[V]{v}^2 \\
          \abs{L(v)} &\le \Gamma \norm[V]{v}
        \end{align*}

        I will show these four conditions in order.
        First, symmetry
        \begin{align*}
          B(u, v) &= \diintt{\Omega}{}{\nabla u \cdot \nabla v}{\v{x}} + \dintt{\partial\Omega}{}{uv}{s} \\
          &= \diintt{\Omega}{}{\nabla v \cdot \nabla u}{\v{x}} + \dintt{\partial\Omega}{}{vu}{s} \\
          &= B(v, u)
        \end{align*}
        Second, boundedness of $B$
        \begin{align*}
          \abs{B(u, v)} &= \abs{\diintt{\Omega}{}{\nabla u \cdot \nabla v}{\v{x}} + \dintt{\partial\Omega}{}{uv}{s}} \\
          &\le \abs{\diintt{\Omega}{}{\nabla u \cdot \nabla v}{\v{x}}} + \abs{\dintt{\partial\Omega}{}{uv}{s}}
          \intertext{Using Cauchy-Schwarz on both integrals gives}
          &\le \norm[L^2(\Omega)]{\norm{\nabla u}} \norm[L^2(\Omega)]{\norm{\nabla v}} + \norm[L^2(\partial\Omega)]{u}\norm[L^2(\partial\Omega)]{v}
          \intertext{Since $\partial\Omega \subset \Omega$, this implies that $\norm[L^2(\partial\Omega)]{u} \le \norm[L^2(\Omega)]{u}$, so}
          &\le \norm[L^2(\Omega)]{\norm{\nabla u}} \norm[L^2(\Omega)]{\norm{\nabla v}} + \norm[L^2(\Omega)]{u}\norm[L^2(\Omega)]{v}
          \intertext{Now since $\norm[H^1(\Omega)]{u}$ is greater than both $\norm[L^2(\Omega)]{\norm{\nabla u}}$ and $\norm[L^2(\Omega)]{u}$}
          \abs{B(u, v)} &\le 2\norm[H^1(\Omega)]{u} \norm[H^1(\Omega)]{v}
        \end{align*}
        Third I will show V-ellipticity of $B$.
        There are two possible cases $v = 0$ on $\partial \Omega$ or $v \neq 0$
        on $\partial \Omega$.
        If $v = 0$ on $\partial \Omega$, then
        \begin{align*}
          B(v, v) &= \diintt{\Omega}{}{\norm{\nabla v}^2}{\v{x}} + \dintt{\partial\Omega}{}{v^2}{s} \\
          &= \diintt{\Omega}{}{\norm{\nabla v}^2}{\v{x}} \\
          &= \frac{1}{2}\p{\diintt{\Omega}{}{\norm{\nabla v}^2}{\v{x}} + \diintt{\Omega}{}{\norm{\nabla v}^2}{\v{x}}} \\
          &= \frac{1}{2}\p{\norm[L^2(\Omega)]{\norm{\nabla v}}^2 + \norm[L^2(\Omega)]{\norm{\nabla v}}^2}
          \intertext{Poincare's Inequality states that there exists a constant
          $C > 0$ such that $C\norm[L^2(\Omega)]{v}^2 \le \norm[L^2(\Omega)]{\norm{\nabla v}}^2$, therefore}
          B(v, v) &\ge \frac{1}{2}\p{\norm[L^2(\Omega)]{\norm{\nabla v}}^2 + C\norm[L^2(\Omega)]{v}^2} \\
          &\ge \frac{1}{2}\min{1, C}\p{\norm[L^2(\Omega)]{\norm{\nabla v}}^2 + \norm[L^2(\Omega)]{v}^2} \\
          &= \frac{1}{2}\min{1, C} \norm[H^1(\Omega)]{v}
        \end{align*}
        If $v \neq 0$ on $\partial \Omega$, then we can make use of Friedrich's
        Inequality, which states that there exists constants $C_1 > 0$ and
        $C_2 > 0$ such that
        \[
          \diintt{\Omega}{}{v^2}{\v{x}} \le C_1\dinntt{\Omega}{}{\norm{\nabla v}^2}{x} + C_2 \dintt{\partial\Omega}{}{v^2}{s}
        \]
        This is equivalent to
        \[
          \frac{1}{C}\diintt{\Omega}{}{v^2}{\v{x}} \le \dinntt{\Omega}{}{\norm{\nabla v}^2}{x} + \dintt{\partial\Omega}{}{v^2}{s}
        \]
        where $C = \max{C_1, C_2}$.
        Now,
        \begin{align*}
          B(v, v) &= \diintt{\Omega}{}{\norm{\nabla v}^2}{\v{x}} + \dintt{\partial\Omega}{}{v^2}{s} \\
          &= \frac{1}{2}\diintt{\Omega}{}{\norm{\nabla v}^2}{\v{x}} + \frac{1}{2}\p{\diintt{\Omega}{}{\norm{\nabla v}^2}{\v{x}} + \dintt{\partial\Omega}{}{v^2}{s}} \\
          \intertext{Using Friedrich's Inequality}
          &= \frac{1}{2}\diintt{\Omega}{}{\norm{\nabla v}^2}{\v{x}} + \frac{1}{2}\p{\diintt{\Omega}{}{\norm{\nabla v}^2}{\v{x}} + \dintt{\partial\Omega}{}{v^2}{s}} \\
        \end{align*}


        Lastly I will show that $L$ is bounded,
        \begin{align*}
          \abs{L(v)} &= \abs{\diintt{\Omega}{}{f v}{\v{x}} + \dintt{\partial \Omega}{}{gv}{s}} \\
          &\le \abs{\diintt{\Omega}{}{f v}{\v{x}}} + \abs{\dintt{\partial \Omega}{}{gv}{s}}
          \intertext{Using Cauchy-Schwarz on both integrals gives}
          &\le \norm[L^2(\Omega)]{f}\norm[L^2(\Omega)]{v} + \norm[L^2(\partial \Omega)]{g}\norm[L^2(\partial \Omega)]{v}
          \intertext{Now since $\norm[H^1(\Omega)]{v}$ is greater than both $\norm[L^2(\Omega)]{v}$ and $\norm[L^2(\partial\Omega)]{v}$}
          &\le \norm[L^2(\Omega)]{f}\norm[H^1(\Omega)]{v} + \norm[L^2(\partial \Omega)]{g}\norm[H^1(\Omega)]{v} \\
          &\le 2\max{\norm[L^2(\Omega)]{f}, \norm[L^2(\partial \Omega)]{g}}\norm[H^1(\Omega)]{v}
        \end{align*}
        This shows that the variational problem satisfies all of the criteria
        for the Lax-Milgram theorem, therefore it has a unique solution.

      \item[(c)]
      \item[(d)]
    \end{enumerate}
  \item[\#2]
\end{enumerate}
\end{document}
