\documentclass[11pt, oneside]{article}
\usepackage[letterpaper, margin=2cm]{geometry}
\usepackage{MATH666}

\begin{document}
\noindent \textbf{\Large{Caleb Logemann \\
MATH666 Finite Element Methods \\
Homework 1
}}

%\lstinputlisting[language=MATLAB]{H01_23.m}
\begin{enumerate}
  \item[\#1]
    \item[(a)] % Done
      In order to recast this as a variational problem we will multiply the
      equation by a test function and integrate.

      \begin{align*}
        -u'' + q u &= f \\
        -\dintt{-\pi}{\pi}{u''v}{x} + \dintt{-\pi}{\pi}{quv}{x} &= \dintt{-\pi}{\pi}{fv}{x}\\
        \dintt{-\pi}{\pi}{u'v'}{x} - \eval*{\p{u'v}}{-\pi}{\pi}+ \dintt{-\pi}{\pi}{quv}{x} &= \dintt{-\pi}{\pi}{fv}{x} \\
        \intertext{If we let $v(\pi) = v(-\pi)$, then the boundary term goes to zero because $u'(\pi) = u'(-\pi)$}
        \dintt{-\pi}{\pi}{u'v'}{x} + \dintt{-\pi}{\pi}{quv}{x} &= \dintt{-\pi}{\pi}{fv}{x} \\
      \end{align*}
      Thus the variational problem is to find a
      $u \in V = \set{\dintt*{-\pi}{\pi}{(u')^2 + qu^2}{x} < \infty | u(\pi) = u(-\pi), u'(\pi) = u'(-\pi)}$
      \begin{align*}
        \dintt{\pi}{\pi}{u'v'}{x} + \dintt{-\pi}{\pi}{quv}{x} &= \dintt{-\pi}{\pi}{fv}{x}
      \end{align*}
      for all $v \in V$.
      The test function $v$ must satisfy the periodic boundary conditions on the
      function value, but I believe we can also choose to have $v$ satisfy the
      periodic condition on the derivative as well.
      This makes the test and trial function spaces the same.

    \item[(b)] % Done
      The energy minimization problem that is equivalent to the variational
      problem is to find a $u \in V$ such that
      \[
        F(u) \le F(w)
      \]
      for all $w \in V$, where
      \[
        F(w) = \frac{1}{2}\p{\dintt{\pi}{\pi}{(w')^2}{x} + \dintt{-\pi}{\pi}{qw^2}{x}} - \dintt{-\pi}{\pi}{fw}{x}
      \]
      I will now prove that the energy minimization problem and the variational
      problem are equivalent.

      \begin{proof}
        Let $u$ be a solution to the variational problem, and consider some
        $w \in V$.
        Then there exists $v \in V$ such that $u + v = w$.
        Now consider $F(w)$.
        \begin{align*}
          F(w) &= F(u + v) \\
          &= \frac{1}{2}\p{\dintt{\pi}{\pi}{(u'+v')^2}{x} + \dintt{-\pi}{\pi}{q(u+v)^2}{x}} - \dintt{-\pi}{\pi}{f(u+v)}{x} \\
          &= \frac{1}{2}\p{\dintt{\pi}{\pi}{(u')^2+2u'v' + (v')^2}{x} + \dintt{-\pi}{\pi}{qu^2+2quv + qv^2}{x}} - \dintt{-\pi}{\pi}{f(u+v)}{x} \\
          &= \frac{1}{2}\p{\dintt{\pi}{\pi}{(u')^2}{x} + \dintt{-\pi}{\pi}{qu^2}{x}} - \dintt{-\pi}{\pi}{fu}{x} + \dintt{-\pi}{\pi}{u'v'}{x} + \dintt{-\pi}{\pi}{quv}{x} - \dintt{-\pi}{\pi}{fv}{x} + \dintt{-\pi}{\pi}{(v')^2 + qv^2}{x} \\
          &= F(u) + \dintt{-\pi}{\pi}{u'v'}{x} + \dintt{-\pi}{\pi}{quv}{x} - \dintt{-\pi}{\pi}{fv}{x} +  \dintt{-\pi}{\pi}{(v')^2 + qv^2}{x} \\
          \intertext{Since $u$ is a solution to the variational problem the middle three terms cancel}
          &= F(u) +  \dintt{-\pi}{\pi}{(v')^2 + qv^2}{x} \\
          \intertext{Also since $q(x) > 0$ for $x \in \br{-\pi, \pi}$, this integral is nonnegative}
          &\ge F(u)
        \end{align*}
        This shows that $u$ is a solution to the energy minization problem when
        $u$ is a solution to the variational problem.

        Now let $u$ be a solution to the energy minimization problem, then 
        \[
          F(u) \le F(w)
        \]
        for all $w \in V$.
        Let $v \in V$ and consider $w = u + \varepsilon v$, then
        $F(u) \le F(u + \varepsilon v)$.
        Consider the function of $\varepsilon$,
        \[
          g(\varepsilon) = F(u + \varepsilon v)
        \]
        we know that $g$ has a minimum at $\varepsilon = 0$ thus $g'(0) = 0$.
        \begin{align*}
          g(\varepsilon) &= \frac{1}{2}\p{\dintt{\pi}{\pi}{(u'+\varepsilon v')^2}{x} + \dintt{-\pi}{\pi}{q(u+\varepsilon v)^2}{x}} - \dintt{-\pi}{\pi}{f(u+\varepsilon v)}{x} \\
          &= \frac{1}{2}\p{\dintt{\pi}{\pi}{(u')^2}{x} + \dintt{-\pi}{\pi}{qu^2}{x}} - \dintt{-\pi}{\pi}{fu}{x} + \varepsilon\dintt{-\pi}{\pi}{u'v'}{x} + \varepsilon\dintt{-\pi}{\pi}{quv}{x} - \varepsilon\dintt{-\pi}{\pi}{fv}{x} + \varepsilon^2\dintt{-\pi}{\pi}{(v')^2 + qv^2}{x} \\
          g'(\varepsilon) &= \dintt{-\pi}{\pi}{u'v'}{x} + \dintt{-\pi}{\pi}{quv}{x} - \dintt{-\pi}{\pi}{fv}{x} + 2\varepsilon\dintt{-\pi}{\pi}{(v')^2 + qv^2}{x} \\
          g'(0) &= \dintt{-\pi}{\pi}{u'v'}{x} + \dintt{-\pi}{\pi}{quv}{x} - \dintt{-\pi}{\pi}{fv}{x} \\
          0 &= \dintt{-\pi}{\pi}{u'v'}{x} + \dintt{-\pi}{\pi}{quv}{x} - \dintt{-\pi}{\pi}{fv}{x} \\
          \dintt{-\pi}{\pi}{fv}{x} &= \dintt{-\pi}{\pi}{u'v'}{x} + \dintt{-\pi}{\pi}{quv}{x} \\
        \end{align*}
        Since this is true for any $v \in V$, this shows that $u$ is a solution
        to the variational problem.
      \end{proof}

    \item[(c)] % Done
      Consider $u, w \in V$ solutions to the variational, then for any $v \in V$
      we have
      \begin{align*}
        \dintt{\pi}{\pi}{u'v'}{x} + \dintt{-\pi}{\pi}{quv}{x} &= \dintt{-\pi}{\pi}{fv}{x} \\
        \dintt{\pi}{\pi}{w'v'}{x} + \dintt{-\pi}{\pi}{qwv}{x} &= \dintt{-\pi}{\pi}{fv}{x} \\
        \dintt{\pi}{\pi}{(u - w)'v'}{x} + \dintt{-\pi}{\pi}{q(u - w)v}{x} &= 0
      \end{align*}
      Thus the Galerkin Orthogonality property of this problem is
      \[
        \dintt{\pi}{\pi}{(u - w)'v'}{x} + \dintt{-\pi}{\pi}{q(u - w)v}{x} = 0
      \]
      for $u$, $w$ solutions to the variational problem and for all $v \in V$.

      The energy norm for this problem is found by letting $v = u - w$ and
      taking the squareroot.
      \[
        \norm[E]{u} = \sqrt{\dintt{\pi}{\pi}{(u')^2}{x} + \dintt{-\pi}{\pi}{qu^2}{x}}
      \]

      Note that an energy inner product can be formed as well,
      \[
        \abr[E]{u, v} = \dintt{\pi}{\pi}{u'v'}{x} + \dintt{-\pi}{\pi}{quv}{x}
        \qquad \norm[E]{u} = \sqrt{\abr[E]{u, u}}
      \]
      This satisfies all the properties of an inner product.
      \begin{align*}
        \abr[E]{u, v} &= \abr[E]{v, u} \\
        \abr[E]{au, v} &= a\abr[E]{u, v} \\
        \abr[E]{u + w, v} &= \abr[E]{u, v} + \abr[E]{w, v} \\
        \abr[E]{u, u} \ge 0 \\
        \abr[E]{u, u} = 0 &\Leftrightarrow u = 0
      \end{align*}
      Note that $q > 0$ is required for the last two statements.
      The fact that this forms an inner product allows the Cauchy-Schwarz
      inequality to be applied directly to the energy norm.
      Also the Galerkin Orthogonality condition can be expressed as
      \[
        \abr[E]{u - w, v} = 0
      \]
      for $u, w$ solutions to the variational problem and for all $v \in V$.

    \item[(d)] % Done
      The cG(1) method for this problem is formulated by replacing the test and
      trial space $V$ with a subspace, $V^1_h$.
      Let $-\pi = x_0 < x_1 < \cdots < x_M < x_{M+1} = \pi$ be a partition of
      $\br{-\pi, \pi}$, and define $h_j = x_{j} - x_{j-1}$ and let
      $h = \max[1 \le j \le M+1]{h_j}$.
      I will also define the functions
      \[
        \phi_j(x) =
        \begin{cases}
          \frac{x - x_{j-1}}{x_j - x_{j-1}} & x_{j-1} \le x \le x_j \\
          \frac{x - x_{j+1}}{x_j - x_{j+1}} & x_{j} \le x \le x_{j+1} \\
          0 & \text{otherwise}
        \end{cases}
      \]
      for $j = 1, 2, \cdots, M$ and
      \[
        \phi_0(x) = \phi_{M+1}(x) =
        \begin{cases}
          \frac{x - x_{0}}{x_1 - x_0} & x_{0} \le x \le x_1 \\
          \frac{x - x_{M+1}}{x_M - x_{M+1}} & x_M \le x \le x_{M+1} \\
          0 & \text{otherwise}
        \end{cases}
      \]

      Then the space $V^1_h = \set{\sum{j = 1}{M+1}{\xi_j \phi_j} | \xi_{M+1} = \frac{h_1 h_{M+1}}{h_1 + h_{M+1}}\p{\frac{\xi_1}{h_1} + \frac{\xi_M}{h_{M+1}}}}$.
      The condition on $\xi_{M+1}$ guarantees the continuity of the derivative
      at $x_0 = x_{M+1}$.
      The design of the basis function $\phi_{M+1}$ guarantees the continuity of
      the function.

    \item[(e)]
      The following will show that the cG(1) method gives the optimal solution
      in the energy norm.

      \begin{proof}
        Let $u$ be the solution to the original variational and let $U$ be the
        solution to the cG(1) method.
        Consider an arbitrary $v \in V^1_h$, and note that Galerkin orthogonality states
        that $\abr[E]{u - U, U - v} = 0$.
        \begin{align*}
          \norm[E]{u - U}^2 &= \abr[E]{u - U, u - U} \\
          &= \abr[E]{u - U, u - U} + \abr[E]{u - U, U - v} \\
          &= \abr[E]{u - U, u - v}
          \intertext{Using Cauchy-Schwarz}
          &\le \norm[E]{u - U}\norm[E]{u - v}
        \end{align*}
        This shows that $\norm[E]{u - U} \le \norm[E]{u - v}$.
      \end{proof}

      Now in order to find an error estimate let $v = \pi_h u$.
      \begin{align*}
        \norm[E]{u - U}^2 &\le \norm[E]{u - \pi_h u}^2 \\
        &= \dintt{-\pi}{\pi}{((u - \pi_h u)')^2}{x} + \dintt{-\pi}{\pi}{q(u - \pi_h u)^2}{x} \\
        &\le C h^2 \norm{u''}^2 + C h^4 \norm{u''}^2
      \end{align*}
      Thus
      \[
        \norm[E]{u - U} \le C h \norm{u''}
      \]
      since the $h$ term dominates the error.

    \item[(f)]


    \item[(g)]

  \item[\#2]

\end{enumerate}
\end{document}
